(10 points)
Assume that $u \in \sobolevhdom{1}$ is a bounded weak solution of
\begin{align*}
  -\sum_{i, j = 1}^n \graddir{j} (a^{i, j} u_{x_i}) = 0
    & \hspace{5mm} x \in \domain
\end{align*}
Show that $u^4$ is a weak sub-solution.

This is a special case of Evans 6.6.11.

First, given a $u \in \sobolevhdom{1}, v \in \cnfuncdom{2} \cap \linffuncdom$, our bilinear form is
\begin{align*}
  B[u, v] = \int_{\domain} \sum_{i, j}^n a^{i, j} u_{x_i} v_{x_j} \dx
\end{align*}
and $u$ satisfies $B[u, v] = 0$ for any $v \in \sobolevhdom{1}$.
We want to show that $B[u^4, v] \leq 0$ for $v \in \sobolevhcsdom{1}$ with $v \geq 0$.
We have
%% We compute for $u^4$:
%% \begin{align*}
%%   -\sum_{i, j = 1}^n \graddir{j} (a^{i, j} \graddir{i} u^4)
%%     = &-4 \sum_{i, j = 1}^n \graddir{j} (a^{i, j} u^3 u_{x_i})
%%     = -4 \sum_{i, j = 1}^n a_{x_j}^{i, j} u^3 u_{x_i} + a^{i, j} u^3 u_{x_i, x_j} + 4 a^{i, j} u^2 u_{x_i} u_{x_j}\\
%%     = &-4 \sum_{i, j = 1}^n u^2 (a_{x_j}^{i, j} u u_{x_i} + a^{i, j} (u u_{x_i, x_j} + 3 u_{x_i} u_{x_j}))
%% \end{align*}
\begin{align*}
  %% 0 \geq &-5 \sum_{i, j = 1}^n u^3 (a_{x_j}^{i, j} u u_{x_i} + a^{i, j} (u u_{x_i, x_j} + 4 u_{x_i} u_{x_j}))
  B[u^4, v] = &\int_{\domain} 4 \sum_{i, j}^n a^{i, j} u^3 u_{x_i} v_{x_j} \dx
    = \int_{\domain} 4 \sum_{i, j}^n a^{i, j} (u_{x_i} (\graddir{j} u^3 v) - 4 u^2 u_{x_i} u_{x_j} v) \dx\\
    = &-12 \int_{\domain} u^2 v \sum_{i, j}^n a^{i, j} u_{x_i} u_{x_j} \dx
 %% \leq &-\int_{\domain} 20 u^3 v \sum_{i, j}^n a^{i, j} \left( \frac{u_{x_i}^2}{2} + \frac{u_{x_j}^2}{2} \right) \dx\\
 %% \leq &-\int_{\domain} 20 u^3 v \left\lvert \grad u \right\rvert^2 \sum_{i, j}^n a^{i, j} \dx
\end{align*}
Then, by the ellipticity of $a$, every term inside the integral is positive, meaning the final result must be non-positive.
Thus, $u^4$ is a weak sub-solution.
%% Then, since $v$ is non-negative, if the sign of the other terms is invariant, we can ignore it in our computation.
%% Then since $u^4 v$ is weakly differentiable and bounded, we must have
%% \begin{align*}
%%   0 = &\int_{\domain} \sum_{i, j}^n 5 a^{i, j} u_{x_i} (\graddir{j} u^4 v) \dx\\
%%   B[u^5, v] = &-\int_{\domain} 20 a^{i, j} u^3 u_{x_i} u_{x_j} v \dx
%% \end{align*}
%% Consider $B[u^4, u]$:
%% \begin{align*}
%%   0 = B[u, u^4] = \int_{\domain} \sum_{i, j}^n a^{i, j} u_{x_i} 4 u^3 u_{x_j} \dx = B[u^4, u]
%% \end{align*}
\newpage
