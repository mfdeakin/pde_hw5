(10 points)
Let $u$ be a smooth function satisfying
\begin{align*}
  -\lapl u + u = f(x), \hspace{5mm} |u| \leq 1 \hspace{10mm} x \in \reals^n
\end{align*}
where
\begin{align*}
  |f(x)| \leq F e^{-|x|}
\end{align*}
Deduce from the maximum principle that $u$ actually decays
\begin{align*}
  |u(x)| \leq C e^{-\frac{1}{2} |x|}
\end{align*}

Since $|u| \leq 1$, for a given radius $R$, we can choose $\epsilon > 0$ s.t. $\epsilon e^{\frac{R}{2}} > 1 > |u|$;
specifically any $\epsilon > e^{-\frac{R}{2}}$ will work.
% For $R > \frac{1}{4}$, let $\epsilon = 4 R e^{-\frac{R}{2}}$.
Define $w = C e^{-\frac{1}{2} \lvert x \rvert} + 4 R e^{-\frac{R}{2}} e^{\frac{1}{2} \lvert x \rvert}$ with this $\epsilon$ and some $C > 0$.
Then we compute
\begin{align*}
  L w = &\epsilon \frac{3 \lvert x \rvert - 2 (n - 1)}{4 \lvert x \rvert} e^{\frac{1}{2} \lvert x \rvert}
         + C \frac{3 \lvert x \rvert + 2 (n - 1)}{4 \lvert x \rvert} e^{-\frac{1}{2} \lvert x \rvert}
\end{align*}
%% and we can choose
%% \begin{align*}
%%   C > &\max \limits_{x \in \reals^n} \frac{4}{3} F e^{-\frac{1}{2} \lvert x \rvert}
%%        + \frac{8}{3} (n - 1) F e^{-\frac{2}{3} (n - 1)}
%% \end{align*}
%% ensuring $L w > F$ when $\lvert x \rvert < R$.
We need $L (w \pm u) \geq L w - F \exp^{-\lvert x \rvert} > 0$.
In the annulus $0 < r = \frac{2}{3}(n - 1) < R < \infty$,
$\epsilon \frac{3 \lvert x \rvert - 2 (n - 1)}{4 \lvert x \rvert} e^{\frac{1}{2} \lvert x \rvert}$
is positive, so consider this to be our domain.
Removing positive terms leaves us with $C > \frac{4}{3} F$ as an easy requirement for $L w > F$ in this annulus.
For the minimum principle to apply, we also need to ensure that $w > 1$ on the boundary $\lvert x \rvert = \frac{2}{3} (n - 1)$,
so choose $C > \max \set{ \frac{4}{3} F, e^{\frac{n - 1}{3}} }$.

Then $w \pm u > 0$ on both boundaries of the annulus,
and $L w > |f| = |L u|$ ensures $L (w \pm u) > 0$ in the annulus, ensuring the minimum principle holds.
Then by the comparison principle, $\lvert u \rvert \leq w$ in $\ball{0}{R}$.
Finally, we note that as $R \to \infty$, we can take $\epsilon$ arbitrarily close to $0$,
proving that the function must decay as $C e^{-\frac{1}{2} \lvert x \rvert}$.
