
\documentclass[letterpaper,11pt]{article}

\usepackage{latexsym}
\usepackage{amsmath}
\usepackage{amssymb}
\usepackage{fancyhdr}
\usepackage[margin=1.0in, left=0.5in, right=0.5in, top=1.25in, headsep=10mm, headheight=15mm]{geometry}
\usepackage{graphicx}

\pagestyle{fancy}
\rhead{Michael Deakin\\Math 516\\Homework 5}

\newcommand*\limitset[1]{{#1}^\prime}
\newcommand*\closure[1]{\overline{#1}}
\newcommand*\closureunion[1]{{#1}\cup \limitset{#1}}
\newcommand*\interior[1]{{#1}^\circ}

% Sets of points
% The set of points within some distance #1 from #2
\newcommand*\neighbor[2]{N_{#1}({#2})}
% The neighborhood without #2
\newcommand*\delneighbor[2]{N_{#1}^*({#2})}
\newcommand*\set[1]{\{ #1 \} }
\newcommand*\conjugate[1]{\overline{#1}}
\newcommand*\sequence[2]{\set{#1}_{#2=1}^\infty}
\newcommand*\series[2]{\sum_{#2=1}^\infty #1_{#2}}
\newcommand*\compose[2]{#1 \circ #2}
\newcommand*\udisk[0]{\mathbb{D}}
\newcommand*\disk[2]{D_{#1}(#2)}
\newcommand*\punctdisk[2]{\disk_{ #1 } - \set{#2}}
\newcommand*\complex[0]{\mathbb{C}}
\newcommand*\naturals[0]{\mathbb{N}}
\newcommand*\rationals[0]{\mathbb{Q}}
\newcommand*\reals[0]{\mathbb{R}}

% Function spaces
\newcommand*\cnfunc[2]{C^{#2}\left(#1 \right)}
\newcommand*\cnfuncdom[1]{C^{#1}\left(\domain \right)}
\newcommand*\linffunc[1]{L^{\infty}\left(#1 \right)}
\newcommand*\linffuncdom[0]{L^{\infty}\left(\domain \right)}
\newcommand*\lnfunc[2]{L^{#2}\left(#1 \right)}
\newcommand*\lnfuncdom[1]{L^{#1}\left(\domain \right)}
\newcommand*\sobolev[3]{W^{#2, #3}\left(#1 \right)}
\newcommand*\sobolevdom[2]{W^{#1, #2}\left(\domain \right)}
\newcommand*\sobolevh[2]{H^{#2}\left(#1 \right)}
\newcommand*\sobolevhdom[1]{H^{#1}\left(\domain \right)}
\newcommand*\sobolevcs[3]{W_0^{#2, #3}\left(#1 \right)}
\newcommand*\sobolevcsdom[2]{W_0^{#1, #2}\left(\domain \right)}
\newcommand*\sobolevhcs[2]{H_0^{#2}\left(#1 \right)}
\newcommand*\sobolevhcsdom[1]{H_0^{#1}\left(\domain \right)}

\newcommand*\grad[0]{D}
\newcommand*\graddir[1]{D^{#1}}
\newcommand*\lapl[0]{\Delta}
\newcommand*\diffquot[1]{D^{#1}}
\newcommand*\diffquotdir[2]{D_{#2}^{#1}}

\newcommand*\domain[0]{U}
\newcommand*\bndry[1]{\partial #1}
\newcommand*\bndrydom[0]{\partial \domain}
\newcommand*\compactcont[0]{\subset \subset} % U \compactcont V \rightarrow U \subset \closure{U} \subset V, where U, V are (open) domains

\newcommand*\ball[2]{B_{#2}(#1)}

\newcommand*\limitto[2]{\lim \limits_{#1 \rightarrow #2}}

\newcommand{\dd}[1]{\;\mathrm{d}#1}
\newcommand{\dx}{\dd{x}}
\newcommand{\dy}{\dd{y}}
\newcommand{\dz}{\dd{z}}
\newcommand{\dr}{\dd{r}}
\newcommand{\ds}{\dd{s}}
\newcommand{\dS}{\dd{S}}
\newcommand{\dt}{\dd{t}}
\newcommand*\pderiv[2]{\frac{\partial #1}{\partial #2}}
\newcommand*\nthpderiv[3]{\frac{\partial^{#3} #1}{\partial #2^{#3}}}
\newcommand*\deriv[2]{\frac{\dd{#1}}{\dd{#2}}}
\newcommand*\nthderiv[3]{\frac{\dd{^{#3} #1}}{\dd{#2^{#3}}}}

% Norms
\newcommand*\linfnorm[2]{\left|\left|#1\right|\right|_{L^{\infty}(#2)}}
\newcommand*\linfnormdom[1]{\left|\left|#1\right|\right|_{L^{\infty}(\domain)}}

\newcommand*\lnorm[3]{\left|\left|#1\right|\right|_{L^{#2}(#3)}}
\newcommand*\lnormdom[2]{\left|\left|#1\right|\right|_{L^{#2}(\domain)}}

\newcommand*\hnorm[3]{\left|\left|#1\right|\right|_{H^{#2}(#3)}}
\newcommand*\hnormdom[2]{\left|\left|#1\right|\right|_{H^{#2}(\domain)}}

\newcommand*\wnorm[4]{\left|\left|#1\right|\right|_{W^{#2, #3}(#4)}}
\newcommand*\wnormdom[3]{\left|\left|#1\right|\right|_{W^{#2, #3}(\domain)}}

\DeclareMathOperator{\res}{res}
\DeclareMathOperator{\sign}{sign}
\DeclareMathOperator{\diam}{diam}
\DeclareMathOperator{\partition}{Partition}

% Average integral from https://tex.stackexchange.com/questions/759/average-integral-symbol
\def\Xint#1{\mathchoice
{\XXint\displaystyle\textstyle{#1}}%
{\XXint\textstyle\scriptstyle{#1}}%
{\XXint\scriptstyle\scriptscriptstyle{#1}}%
{\XXint\scriptscriptstyle\scriptscriptstyle{#1}}%
\!\int}
\def\XXint#1#2#3{{\setbox0=\hbox{$#1{#2#3}{\int}$ }
\vcenter{\hbox{$#2#3$ }}\kern-.6\wd0}}
\def\ddashint{\Xint=}
\def\dashint{\Xint-}
\def\avgint{\dashint}

\begin{document}
I collaborated with Damien Huet on this assignment
\begin{enumerate}
\item (10 points)
Consider the following convection-diffusion problem
\begin{align*}
  \begin{cases}
    -\lapl u + \sum_{j = 1}^{n} b_j \diffquot{j}{} u + c u = f & \text{in } \domain\\
    u = 0 & \text{in } \bndrydom
  \end{cases}
\end{align*}
Assume that $f \in L^2(\domain)$, $b_j \in C^1(\closure{\domain})$,
$c \in \linffuncdom$.
Show that if $c - \frac{1}{2} \sum_{j = 1}^{n} \diffquot{j}{}(b_j) \geq 0$,
then the above problem has a unique weak solutions.

First we define the following bilinear mapping, and apply integration by parts:
\begin{align*}
  B[u, v] = &\int_{\domain} (D u)^T (D v) + v b \cdot (D u) + c u v \dx\\
          = &\int_{\domain} (D u)^T (D v) + \sum_{j = 1}^n v b_j (D_j u) + c u v \dx\\
          = &\int_{\domain} (D u)^T (D v) - \sum_{j = 1}^n (D_j v b_j) u + c u v \dx
\end{align*}

\item (20 points)
Let $\domain$ be a bounded domain in $\reals^2$.
Consider the following minimization problem:
\begin{align*}
  c = &\inf_{u \in \sobolevhcsdom{1}} \left( \frac{1}{2} \int_\domain |\grad u|^2 \dx
                                        + \frac{1}{4} \int_\domain u^4 \dx
                                        - \int_\domain f u \dx
                                 \right)
\end{align*}
Show that $c$ can be attained and its minimizer is a weak solution
\begin{align*}
  \begin{cases}
    \Delta u = u^3  + f(x) & x \in \domain\\
    u = 0 & x \in \bndrydom
  \end{cases}
\end{align*}
Show that the weak solution is also unique.

\item (10 points)
Let $u \in \sobolevh{\reals^n}{1}$ have compact support and be a weak solution
of the semilinear PDE
\begin{align*}
  \lapl u(x) = u^3(x) - f(x) & \hspace{5mm} x \in \reals^n
\end{align*}
where $f \in \lnfunc{\reals^n}{2}$.
Prove that $\lnorm{\grad^2 u}{2}{\reals^n} \leq C \lnorm{f}{2}{\reals^n}$.
(Hint: Mimic the proof of $H^2$ estimates but without the cutoff function)

This is just a special case of Evans problem 6.7.
Following the proof for Theorem 6.3.1, since $u$ is a weak solution to the PDE,
if we choose
$v = -\diffquotdir{-h}{k}(\diffquotdir{h}{k} u)$,
we can use the difference quotient integration by parts formula to compute
\begin{align*}
  \int_{\reals^n} (\grad u) \cdot (\grad v) \dx
       = &\sum_{i = 1}^n \int_{\reals^n} u_{x_i} (-\diffquotdir{-h}{k} \diffquotdir{h}{k} u_{x_i}) \dx
        = \sum_{i = 1}^n \int_{\reals^n} (\diffquotdir{h}{k} u_{x_i})^2 \dx\\
       = &\int_{\reals^n} \left|\diffquotdir{h}{k} \grad u \right|^2 \dx
\end{align*}
Next, since $\sobolevh{\reals^n}{1} = \sobolevhcs{\reals^n}{1}$,
we can bound the $\int_{\reals^n} f v \dx$ term using the Cauchy inequality and
Theorem 3 from Evans 5.8:
\begin{align*}
  \int_{\reals^n} f v \dx \leq &\int_{\reals^n} f v \dx
    \leq \epsilon \int_{\reals^n} f^2 \dx + \frac{1}{4 \epsilon} \int_{\reals^n} v^2 \dx\\
    = &\epsilon \int_{\reals^n} f^2 \dx
       + \frac{1}{4 \epsilon} \int_{\reals^n} (\diffquotdir{-h}{k} \diffquotdir{h}{k} u)^2 \dx\\
 \leq &\epsilon \int_{\reals^n} f^2 \dx
       + \frac{C}{4 \epsilon} \int_{\reals^n} (\grad \diffquotdir{h}{k} u)^2 \dx\\
    = &\epsilon \int_{\reals^n} f^2 \dx
       + \frac{C}{4 \epsilon} \int_{\reals^n} (\diffquotdir{h}{k} \grad u)^2 \dx
\end{align*}
where $C > 0$ is dependent only on $u$.
Then we can choose $\epsilon = \frac{C}{2}$, leaving us with
\begin{align*}
  \int_{\reals^n} f v \dx \leq &\int_{\reals^n} f v \dx
 \leq C \int_{\reals^n} f^2 \dx
       + \frac{1}{2} \int_{\reals^n} (\diffquotdir{h}{k} \grad u)^2 \dx
\end{align*}
To conclude, using the difference quotient integration by parts formula and difference of cubes, we have
\begin{align*}
  \int_{\reals^n} u^3 v \dx = &\int_{\reals^n} u^3 (-\diffquotdir{-h}{k} (\diffquotdir{h}{k} u)) \dx
    = \int_{\reals^n} (\diffquotdir{h}{k} u^3) (\diffquotdir{h}{k} u) \dx\\
    = &\int_{\reals^n} (u^3(x + h e_k) - u^3(x)) (\diffquotdir{h}{k} u) \dx\\
    = &\int_{\reals^n} \frac{1}{h}((u(x + h e_k) - u(x)) (u^2(x + h e_k) + u(x + h e_k) u(x) + u^2(x))) (\diffquotdir{h}{k} u) \dx\\
 \geq &\int_{\reals^n} \left(\frac{1}{2} u^2(x + h e_k) + u(x + h e_k) u(x) + \frac{1}{2} u(x)) \right) (\diffquotdir{h}{k} u)^2 \dx\\
    = &\int_{\reals^n} \frac{1}{2}(u(x + h e_k) + u(x)))^2 (\diffquotdir{h}{k} u)^2 \dx\\
    \geq &0
\end{align*}
Thus,
\begin{align*}
  \int_{\reals^n} \left|\diffquotdir{h}{k} \grad u \right|^2 \dx
    \leq C \int_{\reals^n} f^2 \dx
       + \frac{1}{2} \int_{\reals^n} (\diffquotdir{h}{k} \grad u)^2 \dx
\end{align*}
%% Then recall that $\sobolevh{\reals^n}{1} = \sobolevhcs{\reals^n}{1}$,
%% so we can apply Evans 5.8.2 Theorem 3 to conclude
%% $\int_{\reals^n} (D u) \cdot (D v) \dx \leq \int_{\ \diffquotdir{h}{k}reals^n} (\lapl u)$

\item (10 points)
Assume that $u \in \sobolevhdom{1}$ is a bounded weak solution of
\begin{align*}
  -\sum_{i, j = 1}^n \graddir{j} (a^{i, j} \graddir{i} u) = 0 & & \in \Omega
\end{align*}
Show that $u^5$ is a weak sub-solution

\item (20 points)
Let $u$ be a weak sub-solution of
\begin{align*}
  -\sum_{i, j} \graddir{i}(a^{i, j} \graddir{j} u) + c u = f
\end{align*}
where $\theta \leq a^{i, j} \leq C_2 < \infty$.
Suppose that $c \in \lnfunc{B_1}{\frac{n}{2}}$, $f \in \lnfunc{B_1}{q}$
where $q > \frac{n}{2}$.
Show that there exists a generic constant $\epsilon_0 > 0$ such that if
$\int_{B_1} |c|^{\frac{n}{2}} \dx \leq \epsilon_0$, then
\begin{align*}
  \sup_{B_{\frac{1}{2}}} u^+ \leq C ( \lnorm{u^+}{2}{B_1} + \lnorm{f}{q}{B_1} )
\end{align*}
(Hint: Follow Moser's Iteration procedure)

\item (10 points)
Show that $u = \log\lvert x \rvert \in \sobolevhdom{1}$ where $\domain = B_{1}(0) \subset \reals^3$
and that it is a weak solution of
\begin{align*}
  -\lapl u + c(x) u = 0
\end{align*}
for some $c(x) \in \lnfunc{\domain}{\frac{3}{2}}$,
but that $u$ is not bounded.

First, for some $\delta \in (0, 1)$, let $\domain_{\delta} = \domain - B_{\delta}(0)$; then we compute $\hnormdom{u}{1}$:
\begin{align*}
  \lnorm{u}{2}{\domain_{\delta}}^2 = &\int_{\domain_{\delta}} \log^2(\lvert x \rvert) \dx
    = \left. \frac{\bunitvolumedef}{n^2} \lvert x \rvert^n
        \left( n^2 \log^2(\lvert x \rvert) - 2 n \log(\lvert x \rvert) + 2 \right) \right \rvert_{\lvert x \rvert = \delta}^{1}
      \rightarrow 2 \frac{\bunitvolumedef}{n^2}\\
  u_{x_k} = &\frac{x_k}{\lvert x \rvert^2}\\
  \lnorm{\grad u}{2}{\domain_{\delta}}^2 = &\int_{\domain_{\delta}} \frac{1}{\lvert x \rvert^2} \dx
    = \left. \frac{\bunitsurfareadef}{n - 2} \lvert x \rvert^{n - 2} \right \rvert_{\lvert x \rvert = \delta}^{1}
    = \frac{\bunitsurfareadef}{n - 2} (1 - \delta^{n - 2}) \rightarrow \frac{\bunitsurfareadef}{n - 2}
\end{align*}
as $\delta \rightarrow 0^+$. (Note that we used $n \geq 3$ in the computation of the gradient norm).

Then we have $\hnormdom{u}{1} = 2 \frac{\bunitvolumedef}{n^2} + \frac{\bunitsurfareadef}{n - 2}$.
Since this is finite, $u \in \sobolevhdom{1}$.
We also note that $\limitto{x}{0} u = -\infty$, so $u$ is not bounded.

Next, let $c(x) = \frac{\Delta u}{u} = \frac{2 - n}{\lvert x \rvert^2 \log \lvert x \rvert}$
(this isn't technically in $\lnfunc{\domain}{\frac{3}{2}}$, some modification of the domain and $u$ is needed).
Clearly, $u$ satisfies the strong equation everywhere in $U$ except possibly at $0$ with this $c$.
Then for any $v \in \sobolevhdom{1}$, with $\nu = \frac{x}{\lvert x \rvert}$ we must have
\begin{align*}
  0 = &\int_{\domain_{\delta}} -\lapl u v + c(x) u v \dx
     = \int_{\domain_{\delta}} (\grad u) \cdot (\grad v) + c(x) u v \dx
       - \int_{\bndrydom} v \left( \pderiv{u}{\nu} \right) \dS
       + \int_{\bndry{\ball{0}{\delta}}} v \left( \pderiv{u}{\nu} \right) \dS\\
    = &\int_{\domain_{\delta}} (\grad u) \cdot (\grad v) + c(x) u v \dx
       + \int_{\bndry{\ball{0}{\delta}}} v \left( \pderiv{u}{\nu} \right) \dS\\
\end{align*}
Then, noting that $\lvert x \rvert = \delta$, we compute for $n = 3$
\begin{align*}
  \int_{\bndry{\ball{0}{\delta}}} v \left( \pderiv{u}{\nu} \right) \dS
    = &\int_{\bndry{\ball{0}{\delta}}} v \left( \frac{1}{\delta^3} \sum_{k = 1}^{n} x_k x_k \right) \dS
     = \int_{\bndry{\ball{0}{\delta}}} \frac{v}{\delta} \dS\\
    = &\int_{\theta = 0}^{\pi} \int_{\phi = 0}^{2 \pi} v \delta \sin(\theta) \dd{\phi} \dd{\theta}
\end{align*}
Since $v \in \sobolevhdom{1}$, it's bounded on $\bndry{\ball{0}{\delta}}$,
so by the dominated convergence theorem,
\begin{align*}
  \limitto{\delta}{0} \int_{\bndry{\ball{0}{\delta}}} v \left( \pderiv{u}{\nu} \right) \dS =
    &\int_{\theta = 0}^{\pi} \int_{\phi = 0}^{2 \pi} \limitto{\delta}{0} v \delta \sin(\theta) \dd{\phi} \dd{\theta}
    = 0
\end{align*}
Thus, $u$ is the weak solution for
\begin{align*}
  \int_{\domain} (\grad u) \cdot (\grad v) + \frac{2 - n}{\lvert x \rvert^2 \log \lvert x \rvert}  u v \dx
\end{align*}
for $n = 3$ (and also $n > 3$).

\item (10 points)
Let $u \in \sobolevhcsdom{1}$ be a weak solution of
\begin{align*}
  \begin{cases}
    -\lapl u = |u|^{q - 1} u & x \in \domain\\
    u = 0 & x \in \bndrydom
  \end{cases}
\end{align*}
where $q < \frac{n + 2}{n - 2}$.
Without using Moser's iteration lemma, use the $W^{2, p}$ theory only to show that
$u \in \linffuncdom$.

\item (10 points)
Let $u$ be a smooth solution of $L u = -\sum_{i, j} a^{i, j} u_{x_i x_j} = 0$ in $U$
and $a^{i, j} \in \cnfuncdom{1}$ and uniformly elliptic.
Set $v = |D u|^2 + \lambda u^2$.
Show that
\begin{align*}
  \linfnormdom{\grad u} \leq C (\linfnorm{\grad u}{\bndrydom} + \linfnorm{u}{\bndrydom})
\end{align*}

\item (10 points)
Let $u$ be a smooth function satisfying
\begin{align*}
  -\Delta u + u = f(x) & |u| \leq 1 & x \in \reals^n
\end{align*}
where
\begin{align*}
  |f(x)| \leq C e^{-|x|}
\end{align*}
Deduce from the maximum principle that $u$ actually decays
\begin{align*}
  |u(x)| \leq C e^{-\frac{1}{2} |x|}
\end{align*}

\end{enumerate}
\end{document}
