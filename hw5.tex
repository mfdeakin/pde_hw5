
\documentclass[letterpaper,11pt]{article}

\usepackage{latexsym}
\usepackage{amsmath}
\usepackage{amssymb}
\usepackage{fancyhdr}
\usepackage[margin=1.0in, left=0.5in, right=0.5in, top=1.25in, headsep=10mm, headheight=15mm]{geometry}
\usepackage{graphicx}

\pagestyle{fancy}
\rhead{Michael Deakin\\Math 516\\Homework 5}

\newcommand*\limitset[1]{{#1}^\prime}
\newcommand*\closure[1]{\overline{#1}}
\newcommand*\closureunion[1]{{#1}\cup \limitset{#1}}
\newcommand*\interior[1]{{#1}^\circ}

% Sets of points
% The set of points within some distance #1 from #2
\newcommand*\neighbor[2]{N_{#1}({#2})}
% The neighborhood without #2
\newcommand*\delneighbor[2]{N_{#1}^*({#2})}
\newcommand*\set[1]{\{ #1 \} }
\newcommand*\conjugate[1]{\overline{#1}}
\newcommand*\sequence[2]{\set{#1}_{#2=1}^\infty}
\newcommand*\series[2]{\sum_{#2=1}^\infty #1_{#2}}
\newcommand*\compose[2]{#1 \circ #2}
\newcommand*\udisk[0]{\mathbb{D}}
\newcommand*\disk[2]{D_{#1}(#2)}
\newcommand*\punctdisk[2]{\disk_{ #1 } - \set{#2}}
\newcommand*\complex[0]{\mathbb{C}}
\newcommand*\naturals[0]{\mathbb{N}}
\newcommand*\rationals[0]{\mathbb{Q}}
\newcommand*\reals[0]{\mathbb{R}}

% Function spaces
\newcommand*\cnfunc[2]{C^{#2}\left(#1 \right)}
\newcommand*\cnfuncdom[1]{C^{#1}\left(\domain \right)}
\newcommand*\linffunc[1]{L^{\infty}\left(#1 \right)}
\newcommand*\linffuncdom[0]{L^{\infty}\left(\domain \right)}
\newcommand*\lnfunc[2]{L^{#2}\left(#1 \right)}
\newcommand*\lnfuncdom[1]{L^{#1}\left(\domain \right)}
\newcommand*\sobolev[3]{W^{#2, #3}\left(#1 \right)}
\newcommand*\sobolevdom[2]{W^{#1, #2}\left(\domain \right)}
\newcommand*\sobolevh[2]{H^{#2}\left(#1 \right)}
\newcommand*\sobolevhdom[1]{H^{#1}\left(\domain \right)}
\newcommand*\sobolevcs[3]{W_0^{#2, #3}\left(#1 \right)}
\newcommand*\sobolevcsdom[2]{W_0^{#1, #2}\left(\domain \right)}
\newcommand*\sobolevhcs[2]{H_0^{#2}\left(#1 \right)}
\newcommand*\sobolevhcsdom[1]{H_0^{#1}\left(\domain \right)}

\newcommand*\grad[0]{D}
\newcommand*\graddir[1]{D^{#1}}
\newcommand*\lapl[0]{\Delta}
\newcommand*\diffquot[1]{D^{#1}}
\newcommand*\diffquotdir[2]{D_{#2}^{#1}}

\newcommand*\domain[0]{U}
\newcommand*\bndry[1]{\partial #1}
\newcommand*\bndrydom[0]{\partial \domain}
\newcommand*\compactcont[0]{\subset \subset} % U \compactcont V \rightarrow U \subset \closure{U} \subset V, where U, V are (open) domains

\newcommand*\ball[2]{B_{#2}(#1)}

\newcommand*\limitto[2]{\lim \limits_{#1 \rightarrow #2}}

\newcommand{\dd}[1]{\;\mathrm{d}#1}
\newcommand{\dx}{\dd{x}}
\newcommand{\dy}{\dd{y}}
\newcommand{\dz}{\dd{z}}
\newcommand{\dr}{\dd{r}}
\newcommand{\ds}{\dd{s}}
\newcommand{\dS}{\dd{S}}
\newcommand{\dt}{\dd{t}}
\newcommand*\pderiv[2]{\frac{\partial #1}{\partial #2}}
\newcommand*\nthpderiv[3]{\frac{\partial^{#3} #1}{\partial #2^{#3}}}
\newcommand*\deriv[2]{\frac{\dd{#1}}{\dd{#2}}}
\newcommand*\nthderiv[3]{\frac{\dd{^{#3} #1}}{\dd{#2^{#3}}}}

% Norms
\newcommand*\linfnorm[2]{\left|\left|#1\right|\right|_{L^{\infty}(#2)}}
\newcommand*\linfnormdom[1]{\left|\left|#1\right|\right|_{L^{\infty}(\domain)}}

\newcommand*\lnorm[3]{\left|\left|#1\right|\right|_{L^{#2}(#3)}}
\newcommand*\lnormdom[2]{\left|\left|#1\right|\right|_{L^{#2}(\domain)}}

\newcommand*\hnorm[3]{\left|\left|#1\right|\right|_{H^{#2}(#3)}}
\newcommand*\hnormdom[2]{\left|\left|#1\right|\right|_{H^{#2}(\domain)}}

\newcommand*\wnorm[4]{\left|\left|#1\right|\right|_{W^{#2, #3}(#4)}}
\newcommand*\wnormdom[3]{\left|\left|#1\right|\right|_{W^{#2, #3}(\domain)}}

\DeclareMathOperator{\res}{res}
\DeclareMathOperator{\sign}{sign}
\DeclareMathOperator{\diam}{diam}
\DeclareMathOperator{\partition}{Partition}

% Average integral from https://tex.stackexchange.com/questions/759/average-integral-symbol
\def\Xint#1{\mathchoice
{\XXint\displaystyle\textstyle{#1}}%
{\XXint\textstyle\scriptstyle{#1}}%
{\XXint\scriptstyle\scriptscriptstyle{#1}}%
{\XXint\scriptscriptstyle\scriptscriptstyle{#1}}%
\!\int}
\def\XXint#1#2#3{{\setbox0=\hbox{$#1{#2#3}{\int}$ }
\vcenter{\hbox{$#2#3$ }}\kern-.6\wd0}}
\def\ddashint{\Xint=}
\def\dashint{\Xint-}
\def\avgint{\dashint}

\begin{document}
I collaborated with Damien Huet on this assignment
\begin{enumerate}
\item (10 points)
Consider the following convection-diffusion problem
\begin{align*}
  \begin{cases}
    -\lapl u + \sum_{j = 1}^{n} b_j \diffquot{j}{} u + c u = f & \text{in } \domain\\
    u = 0 & \text{in } \bndrydom
  \end{cases}
\end{align*}
Assume that $f \in L^2(\domain)$, $b_j \in C^1(\closure{\domain})$,
$c \in \linffuncdom$.
Show that if $c - \frac{1}{2} \sum_{j = 1}^{n} \diffquot{j}{}(b_j) \geq 0$,
then the above problem has a unique weak solutions.

First we define the following bilinear mapping, and apply integration by parts:
\begin{align*}
  B[u, v] = &\int_{\domain} (D u)^T (D v) + v b \cdot (D u) + c u v \dx\\
          = &\int_{\domain} (D u)^T (D v) + \sum_{j = 1}^n v b_j (D_j u) + c u v \dx\\
          = &\int_{\domain} (D u)^T (D v) - \sum_{j = 1}^n (D_j v b_j) u + c u v \dx
\end{align*}

\item (20 points)
Let $\domain$ be a bounded domain in $\reals^2$.
Consider the following minimization problem:
\begin{align*}
  c = &\inf_{u \in \sobolevhcsdom{1}} \left( \frac{1}{2} \int_\domain |\grad u|^2 \dx
                                        + \frac{1}{4} \int_\domain u^4 \dx
                                        + \int_\domain f u \dx
                                 \right)
\end{align*}
Show that $c$ can be attained and its minimizer is a weak solution
\begin{align*}
  \begin{cases}
    \lapl u = u^3  + f(x) & x \in \domain\\
    u = 0 & x \in \bndrydom
  \end{cases}
\end{align*}
Show that the weak solution is also unique.

First define the bilinear form $B: (\sobolevhdom{1})^2 \rightarrow \reals$
\begin{align*}
  B[u, v] = &\int_{\domain} -(\grad u) \cdot (\grad v) \dx
\end{align*}
If $u$ is a weak solution, for any $v \in \sobolevhdom{1}$ we have
\begin{align*}
  B[u, v] = &\int_{\domain} u^3 v + f v \dx
\end{align*}
Next, define $i: \sobolevhdom{1} \rightarrow [0, \infty)$ with
\begin{align*}
  i[w] = \frac{1}{2} \int_\domain |\grad w|^2 \dx
         + \frac{1}{4} \int_\domain w^4 \dx
         + \int_\domain f w \dx
\end{align*}
Then for some nonzero $v \in \sobolevhdom{1}$ with $v \in \linffuncdom$, $t \in \reals$,
we compute
\begin{align*}
  i[u + v t] = &\frac{1}{2} \int_\domain |\grad (u + v t)|^2 \dx
                + \frac{1}{4} \int_\domain (u + v t)^4 \dx
                + \int_\domain f (u + v t) \dx\\
  \pderiv{}{t} i[u + v t]
             = &\int_\domain \sum_{i = 1}^n (u_{x_i} + v_{x_i} t) v_{x_i}
                + (u + v t)^3 v + f v \dx\\
             = &\int_\domain (\grad u) \cdot (\grad v) + t \lvert \grad v \rvert^2
                + (u + v t)^3 v + f v \dx\\
             = &\int_\domain (\grad u) \cdot (\grad v) + t \lvert \grad v \rvert^2
                + u^3 v + 3 u^2 v^2 t + 3 u v^3 t^2 + v^4 t^3 + f v \dx\\
  \left. \pderiv{}{t} i[u + v t] \right \rvert_{t = 0}
             = &\int_\domain (\grad u) \cdot (\grad v) + u^3 v + f v \dx
\end{align*}
For critical points of $i$, $\left. \pderiv{}{t} i[u + v t] \right \rvert_{t = 0} = 0$
Thus, critical points of $i$ are weak solutions to the PDE.
Next we show this is a minimum:
\begin{align*}
  \nthpderiv{}{t}{2} i[u + v t]
    = &\pderiv{}{t} \int_\domain t \lvert \grad v \rvert^2
                                + 3 u^2 v^2 t + 3 u v^3 t^2 + v^4 t^3 \dx\\
    = & \int_\domain \lvert \grad v \rvert^2 + 3 u^2 v^2 + 6 u v^3 t + 3 v^4 \dx\\
  \left. \nthpderiv{}{t}{2} i[u + v t] \right\rvert_{t = 0}
    = &\int_\domain \lvert \grad v \rvert^2 + 3 u^2 v^2 + 3 v^4 \dx > 0
\end{align*}
Next, consider two minima of $i$, $u_1$ and $u_2$.
Then for any $v \in \sobolevhdom{1}$,
\begin{align*}
  0 = &\left. \pderiv{}{t} i[u_1 + v t] \right\rvert_{t = 0}
    = \int_\domain (\grad u_1) \cdot (\grad v) + u_1^3 v + f v \dx\\
    = &\left. \pderiv{}{t} i[u_2 + v t] \right\rvert_{t = 0}
    = \int_\domain (\grad u_2) \cdot (\grad v) + u_2^3 v + f v \dx\\
  0 = &\int_\domain (\grad (u_1 - u_2)) \cdot (\grad v) + (u_1^3 - u_2^3) v \dx
\end{align*}
Let $v = u_1 - u_2$, and note that $u_1 - u_2 > 0 \rightarrow u_1^3 - u_2^3 > 0$
and $u_1 - u_2 < 0 \rightarrow u_1^3 - u_2^3 < 0$.
Then $(u_1^3 - u_2^3) v \geq 0$, and $(\grad (u_1 - u_2)) \cdot (\grad v) \geq 0$.
Thus, $u_1 = u_2$, as otherwise
$\int_\domain (\grad (u_1 - u_2)) \cdot (\grad v) + (u_1^3 - u_2^3) v \dx > 0$,
proving that $i$ has only one local minimum, and thus,
if the weak solution exists, it is the minimum and unique.

Alternatively, we can show that $i$ is convex; that is, for any $u_0, u_1 \in \sobolevhdom{1}$,
$t \in [0, 1]$, we have $i[(1 - t) u_0 + t u_1] \leq (1 - t) i[u_0] + t i[u_1]$.
This is done by applying Cauchy's inequality:
\begin{align*}
 0 \geq &i[(1 - t) u_0 + t u_1] - (1 - t) i[u_0] + t i[u_1]\\
   = &\int_{\domain} \frac{1}{2} \left(
         \left\lvert (1 - t) \grad u_0 + t \grad u_1 \right \rvert^2
           - (1 - t) \left\lvert \grad u_0 \right\rvert^2
           - t \left\lvert \grad u_1 \right\rvert^2 \right) \dx\\
     &+ \int_{\domain} \frac{1}{4} \left(
         [(1 - t) u_0 + t u_1]^4
         - (1 - t) u_0^4 - t u_1^4 \right) \dx\\
   \leq &\int_{\domain} \frac{1}{2}
           \left(\left\lvert \grad u_0 \right\rvert^2 \left( (1 - t)^2 + 2 \epsilon (1 - t)^2 - (1 - t) \right) \right)
           + \left(\left\lvert \grad u_1 \right\rvert^2 \left( t^2 + 2 \epsilon t^2 - t \right) \right) \dx\\
        &+ \int_{\domain} \frac{1}{4} \left(
          [(1 - t) u_0 + t u_1]^4 - (1 - t) u_0^4 - t u_1^4 \right) \dx\\
   %% = \int_{\domain} \frac{1}{2} \left \lvert (1 - t) \grad u_0 + t \grad u_1 \right \rvert^2
   %%   + \frac{1}{4} ((1 - t) u_0 + t u_1)^4 + ((1 - t) u_0 + t u_1) \dx\\
   %% = \int_{\domain} &\frac{1}{2} \sum_{k = 1}^n (1 - t) u_{0, x_k}^2 + t u_{1, x_k}^2
   %%        - (1 - t)^2 u_{0, x_k} - 2 t (1 - t) u_{0, x_k} u_{1, x_k} - t^2 u_{1, x_k}^2\\
   %%   + &\frac{1}{4} ((1 - t) u_0 + t u_1)^4 + ((1 - t) u_0 + t u_1) \dx\\
\end{align*}
Choosing $\epsilon$ with $\frac{1 - t}{4 t} < \epsilon < \frac{1 - t}{2 t}$
ensures that the first term is negative.
The second term is negative whenever $(1 - t) (u_0 - u_1)^2 > 0$, which is always true for the range of $t$.
Thus, the energy functional is convex, and has at most one minimum.

Finally, assume that $i$ cannot attain its minimum.
Then $\forall u \in \sobolevhdom{1}$,
$i[u] > c$
\newpage

\item (10 points)
Let $u \in \sobolevh{\reals^n}{1}$ have compact support and be a weak solution
of the semilinear PDE
\begin{align*}
  \lapl u(x) = u^3(x) - f(x) & \hspace{5mm} x \in \reals^n
\end{align*}
where $f \in \lnfunc{\reals^n}{2}$.
Prove that $\lnorm{\grad^2 u}{2}{\reals^n} \leq C \lnorm{f}{2}{\reals^n}$.
(Hint: Mimic the proof of $H^2$ estimates but without the cutoff function)

This is just a special case of Evans problem 6.7.
Following the proof for Theorem 6.3.1, since $u$ is a weak solution to the PDE,
if we choose
$v = -\diffquotdir{-h}{k}(\diffquotdir{h}{k} u)$,
we can use the difference quotient integration by parts formula to compute
\begin{align*}
  \int_{\reals^n} (\grad u) \cdot (\grad v) \dx
       = &\sum_{i = 1}^n \int_{\reals^n} u_{x_i} (-\diffquotdir{-h}{k} \diffquotdir{h}{k} u_{x_i}) \dx
        = \sum_{i = 1}^n \int_{\reals^n} (\diffquotdir{h}{k} u_{x_i})^2 \dx\\
       = &\int_{\reals^n} \left|\diffquotdir{h}{k} \grad u \right|^2 \dx
\end{align*}
Next, since $\sobolevh{\reals^n}{1} = \sobolevhcs{\reals^n}{1}$,
we can bound the $\int_{\reals^n} f v \dx$ term using the Cauchy inequality and
Theorem 3 from Evans 5.8:
\begin{align*}
  \int_{\reals^n} f v \dx \leq &\int_{\reals^n} f v \dx
    \leq \epsilon \int_{\reals^n} f^2 \dx + \frac{1}{4 \epsilon} \int_{\reals^n} v^2 \dx\\
    = &\epsilon \int_{\reals^n} f^2 \dx
       + \frac{1}{4 \epsilon} \int_{\reals^n} (\diffquotdir{-h}{k} \diffquotdir{h}{k} u)^2 \dx\\
 \leq &\epsilon \int_{\reals^n} f^2 \dx
       + \frac{C}{4 \epsilon} \int_{\reals^n} (\grad \diffquotdir{h}{k} u)^2 \dx\\
    = &\epsilon \int_{\reals^n} f^2 \dx
       + \frac{C}{4 \epsilon} \int_{\reals^n} (\diffquotdir{h}{k} \grad u)^2 \dx
\end{align*}
where $C > 0$ is dependent only on $u$.
Then we can choose $\epsilon = \frac{C}{2}$, leaving us with
\begin{align*}
  \int_{\reals^n} f v \dx \leq &\int_{\reals^n} f v \dx
 \leq C \int_{\reals^n} f^2 \dx
       + \frac{1}{2} \int_{\reals^n} (\diffquotdir{h}{k} \grad u)^2 \dx
\end{align*}
To conclude, using the difference quotient integration by parts formula and difference of cubes, we have
\begin{align*}
  \int_{\reals^n} u^3 v \dx = &\int_{\reals^n} u^3 (-\diffquotdir{-h}{k} (\diffquotdir{h}{k} u)) \dx
    = \int_{\reals^n} (\diffquotdir{h}{k} u^3) (\diffquotdir{h}{k} u) \dx\\
    = &\int_{\reals^n} (u^3(x + h e_k) - u^3(x)) (\diffquotdir{h}{k} u) \dx\\
    = &\int_{\reals^n} \frac{1}{h}((u(x + h e_k) - u(x)) (u^2(x + h e_k) + u(x + h e_k) u(x) + u^2(x))) (\diffquotdir{h}{k} u) \dx\\
 \geq &\int_{\reals^n} \left(\frac{1}{2} u^2(x + h e_k) + u(x + h e_k) u(x) + \frac{1}{2} u(x)) \right) (\diffquotdir{h}{k} u)^2 \dx\\
    = &\int_{\reals^n} \frac{1}{2}(u(x + h e_k) + u(x)))^2 (\diffquotdir{h}{k} u)^2 \dx\\
    \geq &0
\end{align*}
Thus,
\begin{align*}
  \int_{\reals^n} \left|\diffquotdir{h}{k} \grad u \right|^2 \dx
    \leq C \int_{\reals^n} f^2 \dx
       + \frac{1}{2} \int_{\reals^n} (\diffquotdir{h}{k} \grad u)^2 \dx
\end{align*}
%% Then recall that $\sobolevh{\reals^n}{1} = \sobolevhcs{\reals^n}{1}$,
%% so we can apply Evans 5.8.2 Theorem 3 to conclude
%% $\int_{\reals^n} (D u) \cdot (D v) \dx \leq \int_{\ \diffquotdir{h}{k}reals^n} (\lapl u)$

\item (10 points)
Assume that $u \in \sobolevhdom{1}$ is a bounded weak solution of
\begin{align*}
  -\sum_{i, j = 1}^n \graddir{j} (a^{i, j} u_{x_i}) = 0
    & \hspace{5mm} x \in \domain
\end{align*}
Show that $u^4$ is a weak sub-solution.

This is a special case of Evans 6.6.11.

First, given a $u \in \sobolevhdom{1}, v \in \cnfuncdom{2} \cap \linffuncdom$, our bilinear form is
\begin{align*}
  B[u, v] = \int_{\domain} \sum_{i, j}^n a^{i, j} u_{x_i} v_{x_j} \dx
\end{align*}
and $u$ satisfies $B[u, v] = 0$ for any $v \in \sobolevhdom{1}$.
We want to show that $B[u^4, v] \leq 0$ for $v \in \sobolevhcsdom{1}$ with $v \geq 0$.
We have
%% We compute for $u^4$:
%% \begin{align*}
%%   -\sum_{i, j = 1}^n \graddir{j} (a^{i, j} \graddir{i} u^4)
%%     = &-4 \sum_{i, j = 1}^n \graddir{j} (a^{i, j} u^3 u_{x_i})
%%     = -4 \sum_{i, j = 1}^n a_{x_j}^{i, j} u^3 u_{x_i} + a^{i, j} u^3 u_{x_i, x_j} + 4 a^{i, j} u^2 u_{x_i} u_{x_j}\\
%%     = &-4 \sum_{i, j = 1}^n u^2 (a_{x_j}^{i, j} u u_{x_i} + a^{i, j} (u u_{x_i, x_j} + 3 u_{x_i} u_{x_j}))
%% \end{align*}
\begin{align*}
  %% 0 \geq &-5 \sum_{i, j = 1}^n u^3 (a_{x_j}^{i, j} u u_{x_i} + a^{i, j} (u u_{x_i, x_j} + 4 u_{x_i} u_{x_j}))
  B[u^4, v] = &\int_{\domain} 4 \sum_{i, j}^n a^{i, j} u^3 u_{x_i} v_{x_j} \dx
    = \int_{\domain} 4 \sum_{i, j}^n a^{i, j} (u_{x_i} (\graddir{j} u^3 v) - 4 u^2 u_{x_i} u_{x_j} v) \dx\\
    = &-12 \int_{\domain} u^2 v \sum_{i, j}^n a^{i, j} u_{x_i} u_{x_j} \dx
 %% \leq &-\int_{\domain} 20 u^3 v \sum_{i, j}^n a^{i, j} \left( \frac{u_{x_i}^2}{2} + \frac{u_{x_j}^2}{2} \right) \dx\\
 %% \leq &-\int_{\domain} 20 u^3 v \left\lvert \grad u \right\rvert^2 \sum_{i, j}^n a^{i, j} \dx
\end{align*}
Then, by the ellipticity of $a$, every term inside the integral is positive, meaning the final result must be non-positive.
Thus, $u^4$ is a weak sub-solution.
%% Then, since $v$ is non-negative, if the sign of the other terms is invariant, we can ignore it in our computation.
%% Then since $u^4 v$ is weakly differentiable and bounded, we must have
%% \begin{align*}
%%   0 = &\int_{\domain} \sum_{i, j}^n 5 a^{i, j} u_{x_i} (\graddir{j} u^4 v) \dx\\
%%   B[u^5, v] = &-\int_{\domain} 20 a^{i, j} u^3 u_{x_i} u_{x_j} v \dx
%% \end{align*}
%% Consider $B[u^4, u]$:
%% \begin{align*}
%%   0 = B[u, u^4] = \int_{\domain} \sum_{i, j}^n a^{i, j} u_{x_i} 4 u^3 u_{x_j} \dx = B[u^4, u]
%% \end{align*}
\newpage

\item (20 points)
Let $u$ be a weak sub-solution of
\begin{align*}
  -\sum_{i, j} \graddir{i} (a^{i, j} \graddir{j} u) + c u = f
\end{align*}
where $\theta \leq a^{i, j} \leq C_2 < \infty$.
Suppose that $c \in \lnfunc{\ballunit}{\frac{n}{2}}$, $f \in \lnfunc{\ballunit}{q}$
where $q > \frac{n}{2}$.
Show that there exists a generic constant $\epsilon_0 > 0$ such that if
$\int_{\ballunit} |c|^{\frac{n}{2}} \dx \leq \epsilon_0$, then
\begin{align*}
  \sup_{B_{\frac{1}{2}}} u^+ \leq C ( \lnorm{u^+}{2}{\ballunit} + \lnorm{f}{q}{\ballunit} )
\end{align*}
(Hint: Follow Moser's Iteration procedure)

This is pretty much the same as in Han-Lin's Ch 4,
except we must use that $\sobolevh{\ballunit}{\frac{n}{2}}$
is only contained in $L^2$, not compactly contained.

The primary difference from Moser's proof of local boundedness is that
$c_0 = \lvert c \rvert + \frac{\lvert f \rvert}{k}$
is not necessarily in $\lnfunc{\ballunit}{q}$.
However, $f$ is in $\lnfunc{\ballunit}{\frac{n}{2}}$.
So we should define $\Lambda$ s.t.
$\linfnorm{a^{i, j}}{\ballunit} + \lnorm{c}{\frac{n}{2}}{\ballunit} \leq \Lambda < \infty$,
and $c_0 = \lvert c \rvert + \frac{\lvert f \rvert}{k}$.
Then $\lnorm{c_0}{\frac{n}{2}}{\ballunit} \leq \Lambda + 1$
Using Holder's inequality, we find
\begin{align*}
  \int c_0 w^2 \eta^2 \leq
    & \left( \int_{\ballunit} c^{\frac{n}{2}} \dx \right)^{\frac{2}{n}}
      \left( \int_{\ballunit} (\eta w)^{\frac{n}{n - 2}} \dx \right)^{\frac{n - 2}{n}}
      \leq (\Lambda + 1) \left( \int_{\ballunit} (\eta w)^{\frac{n}{n - 2}} \dx \right)^{\frac{n - 2}{n}}
\end{align*}

%% START from lecture notes\\

%% Define $w = \overline{u}_m^{\frac{\beta}{2}} \overline{u}$,
%% and let $\eta$ be a smooth mollifier with $\eta(\ball{0}{r}) = \set{ 1 }$,
%% $\eta(\ball{0}{R}^C) = \set{ 0 }$ for some $0 < r < R < 1$,
%% and $\lvert \grad \eta \rvert \leq \frac{C}{R - r}$ for some $C$.

%% We can bound the terms of our weak form with this test function $v$
%% $\int_{\ball{0}{1}} a^{i, j} (\graddir{i} u) (\graddir{j} w \eta^2 + (w - k^{\beta + 1}) 2 (\graddir{j} \eta) \eta \dx$.
%% Starting with the second term, applying ellipticity and properties of $\overline{u}$, we get
%% \begin{align*}
%%   & \int_{\ball{0}{1}} a^{i, j} \graddir{i} u [\graddir{j} w \eta^2 + 2 (w - k^{\beta + 1}) (\graddir{j} \eta) \eta] \dx\\
%%   \geq & \theta \int_{\ball{0}{1}} \beta \eta^2 \overline{u}_m^\beta \lvert \grad \overline{u}_m \rvert^2
%%            + \eta^2 \overline{u}_m^\beta \lvert \grad \overline{u} \rvert^2 \dx\\
%%   & \lvert \grad w \rvert^2 \leq & \overline{u}_m^{\beta} \overline{u}^2 \lvert \grad \overline{u}_m \rvert^2
%%     + \overline{u}_m^\beta \lvert \grad \overline{u} \rvert^2\\
%%     \leq & \overline{u}_m^{\beta - 1} \overline{u} \lvert \grad \overline{u}_m \rvert^2
%%            + \overline{u}_m^{\beta} \lvert \grad \overline{u} \rvert^2\\
%%   \int_{\ball{0}{1}} a^{i, j} \graddir{i} u [\graddir{j} w \eta^2 + 2 (w - k^{\beta + 1}) (\graddir{j} \eta) \eta] \dx\\
%% \end{align*}

%% We bound the

%% For large $k$, we have
%% \begin{align*}
%%   \int \lvert \grad (w \eta) \rvert^2 \dx
%%     \leq & C \int \lvert w \rvert^2 ( \lvert \nabla \grad \eta \rvert^2 + \eta^2 ) \dx
%% \end{align*}
%% which we can apply the Sobolev inequality to get
%% \begin{align*}
%%   \lnorm{w \eta}{\frac{2 n}{n - 2}}{\ball{0}{1}}^2
%%     \leq & C \lnorm{w \eta}{2}{\ball{0}{1}}^2
%%       \int \lvert \grad (w \eta) \rvert^2 \dx
%%     \leq C \int \lvert w \rvert^2 ( \lvert \nabla \grad \eta \rvert^2 + \eta^2 ) \dx\\
%%     \leq & \frac{C}{R - r} \int_{\ball{0}{R}} \lvert w \rvert^2 \dx
%% \end{align*}

%% START from Han, Lin\\

First, if $\lnorm{f}{q}{\ballunit} > 0$, let $k = \lnorm{f}{\frac{n}{2}}{\ballunit}$,
otherwise let it be arbitrarily larger than $0$.
Then for some $m > 0$, $\beta \geq 0$, with mollifier $\eta$ with $\eta(\ball{0}{r}) = \set{ 1 }$,
$\eta(\ball{0}{R}^C) = \set{ 0 }$ for some $0 < r < R < 1$,
and $\lvert \grad \eta \rvert \leq \frac{C}{R - r}$ for some $C$, define
\begin{align*}
  \overline{u} = &u^+ + k\\
  \overline{u}_m = &
  \begin{cases}
    \overline{u} & \text{if } u < m\\
    k + m & \text{if } u \geq m\\
  \end{cases}\\
  v = &\eta^2 (\overline{u}_{m}^{\beta} \overline{u} - k^{\beta + 1})
         \in \sobolevhcs{\ballunit}{1}\\
 \grad v = &\eta^2 \overline{u}_m^\beta (\beta \grad \overline{u}_m + \grad \overline{u})
            + 2 \eta \grad \eta (\overline{u}_m^\beta \overline{u} - k^{\beta + 1})\\
\end{align*}
By the ellipticity of $a$, existence of $\lnorm{c}{\frac{n}{2}}{\ballunit}$, we have
\begin{align*}
  \int_{\domain} a^{i, j} (\graddir{i} u) (\graddir{j} v)
    \geq & \theta \int_{\domain} \beta \eta^2 \overline{u}_m^\beta \lvert \grad \overline{u}_m \rvert^2
           + \eta^2 \overline{u}_m^\beta \lvert \grad \overline{u} \rvert^2 \dx
           - \Lambda \int_{\domain} \lvert \grad \overline{u} \rvert \lvert \grad \eta \rvert \overline{u}_m^\beta \overline{u} \eta \dx\\
    \geq & \theta \int_{\domain} \beta \eta^2 \overline{u}_m^\beta \lvert \grad \overline{u}_m \rvert^2
           + \frac{1}{2} \eta^2 \overline{u}_m^\beta \lvert \grad \overline{u} \rvert^2 \dx
           - \frac{2 \Lambda^2}{\theta} \int_{\domain} \lvert \grad \eta \rvert^2 \overline{u}_m^\beta \overline{u}^2 \dx\\
\end{align*}
since % by the Holder inequality
\begin{align*}
  \Lambda \int_{\domain} \lvert \grad \overline{u} \rvert \lvert \grad \eta \rvert \overline{u}_m^\beta \overline{u} \eta \dx
    \leq & \frac{\theta}{2} \int_{\domain} \eta^2 \overline{u}_m^\beta \lvert \grad \overline{u} \rvert^2 \dx
           + \frac{2 \Lambda^2}{\theta} \int_{\domain} \lvert \grad \eta \rvert^2 \overline{u}_m^\beta \overline{u}^2 \dx
\end{align*}
We are able to bound the first two terms by making use of $\overline{u} \geq k$:
\begin{align*}
  \int_{\domain} \beta \eta^2 \overline{u}_m^\beta \lvert \grad \overline{u}_m \rvert^2
      + \eta^2 \overline{u}_m^\beta \lvert \grad \overline{u} \rvert^2 \dx
    \leq & C \int_{\domain} \lvert \grad \eta \rvert^2 \overline{u}_m^\beta \overline{u}^2
      + \left( \lvert c \rvert + \frac{\lvert f \rvert}{k} \right) \eta^2 \overline{u}_m^\beta \overline{u}^2
\end{align*}
Let $c_0 = \lvert c \rvert + \frac{\lvert f \rvert}{k}$,
with $\lnorm{c_0}{\frac{n}{2}}{\ball{0}{1}} \leq \Lambda + 1$.

Then
\begin{align*}
  \int_{\ball{0}{r}} \lvert \grad w \rvert^2 \eta^2 \dx
    \leq C (1 + \beta) \int_{\ball{0}{1}} w^2 \eta^2 + w^2 \lvert \grad \eta \rvert^2 \dx
\end{align*}

Using Holder's inequality, we have
\begin{align*}
  \int_{\ball{0}{1}} c_0 \lvert \grad w \rvert^2 \eta^2 \dx
    \leq & \left( \int_{\ball{0}{1}} c_0^{\frac{n}{2}} \dx \right)^{\frac{2}{n}}
           \left( \int_{\ball{0}{1}} (\eta w)^{\frac{2 n}{n - 2}} \dx \right)^{\frac{n - 2}{n}}\\
    \leq & (\Lambda + 1)
           \left( \int_{\ball{0}{1}} (\eta w)^{\frac{2 n}{n - 2}} \dx \right)^{\frac{n - 2}{n}}
\end{align*}
We compute, using properties of $\overline{u}_m$ and $\eta$, the Sobolev inequality gives
\begin{align*}
  \lnorm{\eta w}{\frac{2 n}{n - 2}}{\ballunit} \leq &
    \lnorm{\eta w}{\frac{2 n}{n - 2}}{\ballunit}
    % + C \epsilon^{-\frac{n}{2 q - n}} \lnorm{\eta w}{2}{\ballunit}\\
    \leq \lnorm{\grad (\eta w)}{2}{\ballunit}
    % + C \epsilon^{-\frac{n}{2 q - n}} \lnorm{\eta w}{2}{\ballunit}
\end{align*}
% for small positive $\epsilon$.
Note the difference from the more general problem in class,
since we can't take the L-$q$ norm.

The Key Step: Recall that $\lnfuncdom{b} \subset \lnfuncdom{a}$ when $b \geq a$.
We then compute (since the Sobolev inequality works in the equality case, but not compactly.)
\begin{align*}
  \lnorm{\lvert \grad \eta w \rvert}{2}{\ballunit}^2
    \leq & C \int_{\ball{0}{1}} (1 + \beta) w^2 \lvert \grad \eta \rvert^2
           + (1 + \beta)^{\frac{2 q}{2 q - n}} w^2 \eta^2 \dx
    \leq C (1 + \beta)^{\alpha} \int (\lvert \grad \eta \rvert^2 + \eta^2) w^2
\end{align*}
Applying the Sobolev inequality again gives lets us use this to bound the $\chi$ norm of $\eta w$:
\begin{align*}
  \lnorm{w}{\chi}{\ball{0}{r}}^2
    \leq & C \int_{\ball{0}{1}} (1 + \beta)^\alpha w^2 (\lvert \grad \eta \rvert^2 + \eta^2) \dx\\
    \leq & C \frac{(1 + \beta)^\alpha}{(R - r)^2} \int_{\ball{0}{R}} w^2 \dx\\
  \lnorm{\overline{u}}{\chi \beta}{\ball{0}{r}}^\beta
    \leq & C \frac{(1 + \beta)^\alpha}{(R - r)^2} \int_{\ball{0}{R}} \overline{u}^{\beta + 2} \dx\\
  \lnorm{\overline{u}}{\chi \beta}{\ball{0}{r}}
    \leq & C \left( \frac{(1 + \beta)^\alpha}{(R - r)^2} \right)^{\frac{1}{\beta + 2}} \lnorm{\overline{u}}{\beta + 2}{\ball{0}{R}} \dx\\
    \leq & C \left( \frac{(1 + \beta)^\alpha}{(R - r)^2} \right)^{\frac{1}{\beta + 2}} \lnorm{u^+ + k}{\beta + 2}{\ball{0}{R}} \dx\\
    \leq & C \left( \frac{(1 + \beta)^\alpha}{(R - r)^2} \right)^{\frac{1}{\beta + 2}} \lnorm{u^+ + f}{\beta + 2}{\ball{0}{R}} \dx
\end{align*}
We can then iterate, choose $\beta_i = 2 \chi_i - 2$, and $r_i = \frac{1}{2} + \frac{1}{2^{i + 1}}$.
We end with $\beta_i + 2 \to \infty$, and $r_i \to \frac{1}{2}$, completing the proof,
as each iteration shows
\begin{align*}
  \lnorm{u^+}{\beta_{i + 1} + 2}{\ball{0}{r_{i + 1}}} \leq C ( \lnorm{u^+}{\beta_{i} + 2}{\ball{0}{r_i}} + \lnorm{f}{q}{\ballunit} )
\end{align*}

I'm concerned that this might use the compactly contained sobolev inequality which doesn't exist for this $c$?

\item (10 points)
Show that $u = \log|x| \in \sobolevh{B_1}{1}$ where $B_1 = B_1(0) \subset \reals^3$
and that it is a weak solution of
\begin{align*}
  -\Delta u + c(x) u = 0
\end{align*}
for some $c(x) \in \lnfunc{B_1}{\frac{3}{2}}$,
but that $u$ is not bounded.

\item (10 points)
Let $u \in \sobolevhcsdom{1}$ be a weak solution of
\begin{align*}
  \begin{cases}
    -\lapl u = |u|^{q - 1} u & x \in \domain\\
    u = 0 & x \in \bndrydom
  \end{cases}
\end{align*}
where $q < \frac{n + 2}{n - 2}$.
Without using Moser's iteration lemma, use the $W^{2, p}$ theory only to show that
$u \in \linffuncdom$.

\item (10 points)
Let $u$ be a smooth solution of $L u = -\sum_{i, j} a^{i, j} u_{x_i x_j} = 0$ in $U$
and $a^{i, j} \in \cnfunc{\overline \domain}{1}$ and uniformly elliptic.
Set $v = |D u|^2 + \lambda u^2$.
Show that for large $\lambda$,
\begin{align*}
  L v \leq 0 \hspace{5mm} \in \domain
\end{align*}

Then deduce by the maximum principle that
\begin{align*}
  \linfnormdom{\grad u} \leq C (\linfnorm{\grad u}{\bndrydom} + \linfnorm{u}{\domain})
\end{align*}

First we compute for $v$:
\begin{align*}
  v = &\sum_{k = 1}^{n} u_{x_k}^2 + \lambda u^2\\
  v_{x_i} = &2 \sum_{k = 1}^{n} u_{x_k} u_{x_i, x_k} + 2 \lambda u u_{x_k}\\
  v_{x_i, x_j} = &2 \sum_{k = 1}^{n} (u_{x_i, x_k} u_{x_j, x_k} + u_{x_k} u_{x_i, x_j, x_k}) + 2 \lambda (u_{x_i} u_{x_j} + u u_{x_i, x_k})\\
  L v &= -2 \sum_{i, j = 1}^{n} a^{i, j}
            \left[ \sum_{k = 1}^{n} (u_{x_i, x_k} u_{x_j, x_k} + u_{x_k} u_{x_i, x_j, x_k})
                   + \lambda (u_{x_i} u_{x_j} + u u_{x_i, x_j})
            \right]
\end{align*}
Then using the ellipticity of the operator and that $u$ is a solution to $L u = 0$,
we have some positive $\theta$ s.t.
\begin{align*}
  L v %\leq &  some intermediate calculations here...
      \leq &-2 \theta \left( \sum_{k} |\grad u_{x_k}|^2 + \lambda |\grad u|^2 \right) - 2 \sum_{i, j, k} a^{i, k} u_{x_k} u_{x_i, x_j, x_k}\\
      %= &-2 \theta \left( \sum_{k} |\grad u_{x_k}|^2 + \lambda |\grad u|^2 \right) - 2 \sum_{i, j, k} a^{i, k} u_{x_k} u_{x_i, x_j, x_k}
  \sum_{i, j, k = 1}^{n} a^{i, k} u_{x_k} u_{x_i, x_j, x_k}
    = &\sum_{i, j, k = 1}^{n} u_{x_k} \left[ (a^{i, j} u_{x_i, x_j})_{x_k} - a^{i, j}_{x_k} u_{x_i, x_j} \right]\\
    = &-\sum_{i, j, k = 1}^{n} a^{i, j}_{x_k} u_{x_k} u_{x_i, x_j}\\
 \leq &\\
  L v \leq &-2 \theta \left( \sum_{k} |\grad u_{x_k}|^2 + \lambda |\grad u|^2 \right) + 2 \sum_{i, j, k = 1}^{n} a^{i, j}_{x_k} u_{x_k} u_{x_i, x_j}\\
  \leq &-2 \theta \left( \sum_{k} |\grad u_{x_k}|^2 + \lambda |\grad u|^2 \right) + 2 \sum_{i, j, k = 1}^{n} a^{i, j}_{x_k} \left(\epsilon u_{x_k}^2 + \frac{u_{x_i, x_j}}{4 \epsilon}\right)\\
\end{align*}
Then, since (I assume) $a^{i, j}_{x_k}$ is bounded, choosing a large enough $\epsilon$ gives
$2 \sum_{i, j, k = 1}^{n} a^{i, j}_{x_k} \frac{u_{x_i, x_j}}{4 \epsilon} - 2 \theta \sum_{k} |\grad u_{x_k}|^2 < 0$.
Then for large enough $\lambda$ we have $2 \sum_{i, j, k = 1}^{n} a^{i, j}_{x_k} u_{x_i} - 2 \lambda \theta \sum_{k} |\grad u|^2 < 0$.
Thus, $L v < 0$ in $\domain$, so the strong maximum principle applies to $v$,
and $\max_{\domain} |\grad u|^2 + \lambda u^2 \leq \max_{\bndrydom} |\grad u|^2 + \lambda u^2 = \max_{\bndrydom} |\grad u|^2$,
completing the proof.

\item (10 points)
Let $u$ be a smooth function satisfying
\begin{align*}
  -\Delta u + u = f(x) & |u| \leq 1 & x \in \reals^n
\end{align*}
where
\begin{align*}
  |f(x)| \leq C e^{-|x|}
\end{align*}
Deduce from the maximum principle that $u$ actually decays
\begin{align*}
  |u(x)| \leq C e^{-\frac{1}{2} |x|}
\end{align*}

\end{enumerate}
\end{document}
