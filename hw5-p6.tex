(10 points)
Show that $u = \log\lvert x \rvert \in \sobolevhdom{1}$ where $\domain = B_{1}(0) \subset \reals^3$
and that it is a weak solution of
\begin{align*}
  -\lapl u + c(x) u = 0
\end{align*}
for some $c(x) \in \lnfunc{\domain}{\frac{3}{2}}$,
but that $u$ is not bounded.

First, for some $\delta \in (0, 1)$, let $\domain_{\delta} = \domain - B_{\delta}(0)$; then we compute $\hnormdom{u}{1}$:
\begin{align*}
  \lnorm{u}{2}{\domain_{\delta}}^2 = &\int_{\domain_{\delta}} \log^2(\lvert x \rvert) \dx
    = \left. \frac{\bunitvolumedef}{n^2} \lvert x \rvert^n
        \left( n^2 \log^2(\lvert x \rvert) - 2 n \log(\lvert x \rvert) + 2 \right) \right \rvert_{\lvert x \rvert = \delta}^{1}
      \rightarrow 2 \frac{\bunitvolumedef}{n^2}\\
  u_{x_k} = &\frac{x_k}{\lvert x \rvert^2}\\
  \lnorm{\grad u}{2}{\domain_{\delta}}^2 = &\int_{\domain_{\delta}} \frac{1}{\lvert x \rvert^2} \dx
    = \left. \frac{\bunitsurfareadef}{n - 2} \lvert x \rvert^{n - 2} \right \rvert_{\lvert x \rvert = \delta}^{1}
    = \frac{\bunitsurfareadef}{n - 2} (1 - \delta^{n - 2}) \rightarrow \frac{\bunitsurfareadef}{n - 2}
\end{align*}
as $\delta \rightarrow 0^+$. (Note that we used $n \geq 3$ in the computation of the gradient norm).

Then we have $\hnormdom{u}{1} = 2 \frac{\bunitvolumedef}{n^2} + \frac{\bunitsurfareadef}{n - 2}$.
Since this is finite, $u \in \sobolevhdom{1}$.
We also note that $\limitto{x}{0} u = -\infty$, so $u$ is not bounded.

Next, let $c(x) = \frac{\Delta u}{u} = \frac{2 - n}{\lvert x \rvert^2 \log \lvert x \rvert}$
(this isn't technically in $\lnfunc{\domain}{\frac{3}{2}}$, some modification of the domain and $u$ is needed).
Clearly, $u$ satisfies the strong equation everywhere in $U$ except possibly at $0$ with this $c$.
Then for any $v \in \sobolevhdom{1}$, with $\nu = \frac{x}{\lvert x \rvert}$ we must have
\begin{align*}
  0 = &\int_{\domain_{\delta}} -\lapl u v + c(x) u v \dx
     = \int_{\domain_{\delta}} (\grad u) \cdot (\grad v) + c(x) u v \dx
       - \int_{\bndrydom} v \left( \pderiv{u}{\nu} \right) \dS
       + \int_{\bndry{\ball{0}{\delta}}} v \left( \pderiv{u}{\nu} \right) \dS\\
    = &\int_{\domain_{\delta}} (\grad u) \cdot (\grad v) + c(x) u v \dx
       + \int_{\bndry{\ball{0}{\delta}}} v \left( \pderiv{u}{\nu} \right) \dS\\
\end{align*}
Then, noting that $\lvert x \rvert = \delta$, we compute for $n = 3$
\begin{align*}
  \int_{\bndry{\ball{0}{\delta}}} v \left( \pderiv{u}{\nu} \right) \dS
    = &\int_{\bndry{\ball{0}{\delta}}} v \left( \frac{1}{\delta^3} \sum_{k = 1}^{n} x_k x_k \right) \dS
     = \int_{\bndry{\ball{0}{\delta}}} \frac{v}{\delta} \dS\\
    = &\int_{\theta = 0}^{\pi} \int_{\phi = 0}^{2 \pi} v \delta \sin(\theta) \dd{\phi} \dd{\theta}
\end{align*}
Since $v \in \sobolevhdom{1}$, it's bounded on $\bndry{\ball{0}{\delta}}$,
so by the dominated convergence theorem,
\begin{align*}
  \limitto{\delta}{0} \int_{\bndry{\ball{0}{\delta}}} v \left( \pderiv{u}{\nu} \right) \dS =
    &\int_{\theta = 0}^{\pi} \int_{\phi = 0}^{2 \pi} \limitto{\delta}{0} v \delta \sin(\theta) \dd{\phi} \dd{\theta}
    = 0
\end{align*}
Thus, $u$ is the weak solution for
\begin{align*}
  \int_{\domain} (\grad u) \cdot (\grad v) + \frac{2 - n}{\lvert x \rvert^2 \log \lvert x \rvert}  u v \dx
\end{align*}
for $n = 3$ (and also $n > 3$).
